\section{Design}
\label{Design}

%Opening text...  Typical length: 3-5 pages.
%
%Hardest section to write.  A lot of possible interdependencies.
%
%If you find that you have to have a ``forward'' reference to a
%section of text you've not described yet, it usually means that the
%structure of your paper is wrong.  So avoid fwd refs.  Backward
%references to previous sections is ok, as long as it's not too far in
%the beginning of the paper.
%
%Do an outline of design even print a Table of Contents
%
%Opening: tell reader what to expect.
%
%Open with key design goals, in descending importance.
%
%General rule: whenever you LIST 2 or more items, THINK about their
%order (should it be importance order? chronological?  categorical?)
%
%(A) What are your design goals, and what do they get you?  Separate
%goals with HOW you achieve them.  Possible goals can include:
%
%- improved performance
%
%- improved scalability (same as perf.  but need to test "multiple"
%machines)
%
%- better energy consumption
%
%- improved security (hard to prove "better" security)
%
%- versatility: has more functionality that can be utilized in more
%settings.  A generalization of past specific work.
%
%- compatibility: works with many existing systems, possibly
%unmodified (or with few modifications).
%
%- other design goals?
%
%(B) briefly describe HOW you would accomplish each of your design
%goals.
%
%(C) Show a high-level architectural figure whole system, and describe
%every "box" of section of the figure.
%
%Start with high-level detail of each components, then go into greater
%detail.
%
%(D) Bulk of design: go over every design goal and architectural
%component, and describe it in detail.
%
%Key: don't just say WHAT you did, but WHY you did that.  WHY, WHY,
%WHY!
%
%Tense: past tense for what was designed, present tense to describe
%system operation.  Switch b/t past and present consistently.  NO
%future tense!
%
%-----------------------------------------------------------------------------

\paragraph{Experimental Setup}
We have used an Intel machine with SandyBridge EP processor running
Ubuntu 14.04.2 LTS with Linux kernel 3.13.0-49-generic.  CPU and
memory details are as follows:

\begin{table}[th]
\begin{center} %\caption {CPU and Memory details}\\
    \begin{tabular}{| l | l |}
    \hline
    Architecture & Intel x86\_64\\ \hline
    CPU(s) & 32\\ \hline
    Thread(s) per core & 2\\ \hline
	Core(s) per socket & 8\\ \hline
	Socket(s) & 2\\ \hline
	NUMA node(s) & 2\\ \hline
	Stepping & 7\\ \hline
	CPU MHz & 1200.00\\ \hline
	L1d cache & 32K\\ \hline
	L1i cache & 32K\\ \hline
	L2 cache & 256K\\ \hline
	L3 cache & 20480K\\ \hline
	NUMA node0 CPU(s) & 0-7,16-23\\ \hline
	NUMA node1 CPU(s) & 8-15,24-31\\ \hline
	MemTotal & 32893524 kB\\ \hline
    \end{tabular}
\end{center}
\caption{\capfont CPU and Memory details}
\label{tab:Table1}
\end{table}

\paragraph{Dataset}
Dataset used in the experiment were collected from Florida
Sparse Matrix collection in Matrix Market format.  To ensure a
consistent benchmarking environment, all datasets were verified for
directed graph having a single strongly connected component.  We also
have tried to include some data sets used in the papers being benchmarked
Our datasets vary in size and type to exhaustively cover a wide
range of real world input.

\begin{table}[th]
\begin{center} %\caption {Dataset details}\\
    \begin{tabular}{| l | l | l | l |}
    \hline
	Dataset & Type & N & M\\ \hline
    	gre\_1107 & Small & 1107 & 4557\\ \hline
	Cell1 & Small & 7055 & 27800\\ \hline
	Appu & Small & 14000 & 1.839M\\ \hline

	Conf6\_0-8x8-80 & Medium & 49152 & 1.917M\\ \hline
	Com-dblp  & Medium & 317080 & 2.1M\\ \hline
	Com-amazon & Medium & 334863 & 1.851M\\ \hline
	Fem\_hifreq\_circuit & Medium & 491100 & 19.748M\\ \hline
	Chevron4 & Medium & 711450 & 5664962\\ \hline

	Cage14 & Large & 1.506M & 25.625M\\ \hline
	Cage15 & Large & 5.155M & 94.044M\\ \hline
	Delaunay-n24 & Large & 16.777M & 100.663M\\ \hline
    \end{tabular}
\end{center}
\caption{\capfont Graph Dataset details}
\label{tab:Table2}
\end{table}

\paragraph{Preprocessing}
We wrote converters for each of the algorithms and converted datasets
form Matrix Market format to the accepted format for the algorithms.
To ensure the correctness of each algorithm we verified that all
vertices are being explored.

\paragraph{Algorithms used in Benchmarking}
After our detailed literature survey we shortlisted \emph{six} BFS
implementations.  Idea was to accomodate some latest work along with some
highly efficient BFS algorithms and see if they are energy efficient.
\begin{itemize}[leftmargin=*]
\item \textbf{Algorithm1}
\emph{Avoiding locks and atomic instructions in shared-memory parallel
BFS using optimistic parallelization} ~\cite{WSLDQ-BFS}.  We used WSLDQ (work
stealing, lock free, scale free) implementation suggested in the
paper.
\item \textbf{Algorithm2}
\emph{Efficient Parallel Graph Exploration on Multi-Core CPU and GPU}
~\cite{STANFORD-BFS}.
\item \textbf{Algorithm3}
\emph{A Work-Efficient Parallel Breadth-First Search Algorithm}
~\cite{MIT-BFS}.
\item \textbf{Algorithm4}
\emph{Direction-Optimizing Breadth-First Search} ~\cite{LIGRA-BFS}.
Shared memory C++ implemented, integrated into ligra tool.
\item \textbf{Algorithm5}
Theoretically Optimized BFS, version 1 by (Jesmin J. Tithi, SBU).
\item \textbf{Algorithm6}
Theoretically Optimized BFS, version 2 by (Jesmin J. Tithi, SBU).
\end{itemize}

\paragraph{Tools used in Benchmarking}
We used easy to use command line tools provided as a part of likwid
project. For our benchmarking purposes we used the following:
\begin{itemize}[leftmargin=*]
\item \textbf{LikwidPerfCtr}
It is a simple lightweight command line tool to measure hardware 
performance counters end to end. The Linux msr module provides
an interface to access model specific registers from user space.
The tool uses the same to read out hardware
performance counters with an unmodified linux kernel. Out of the
various pre-configured event sets, called performance groups, with
useful pre-selected event sets and derived metrics, we used the 
following:
\begin{itemize}[leftmargin=*]
\item MEM: The event gives Main Memory's Read, Write \& Total 
Bandwidth in MBytes/sec. It also gives the Datavolume in GBytes.
\item L2CACHE: The event gives L2 cache miss rate/ratio.
\item L3CACHE: The event primarily gives L3 cache miss rate/ratio.
\end{itemize}
\emph{Command used: likwid-perfctr -c <cores> -g <event> -M <type-of-access> -O} 
\item \textbf{LikwidPowermeter}
Intel recently introduced an interface to configure and readout 
energy consumption of processors and memory. This so called RAPL 
interface is controlled through MSR registers. likwid-powermeter 
is a small tool which allows you to query the energy consumed within 
a package for a given time period and computes the resulting power 
consumption.
\end{itemize}


\paragraph{Experiment readings}
We have explored different MEM and CPU parameters which can possibly
affect energy efficienty.  We have run the experiments with different
combination of CILK and NUMA parameters like \emph{CILK\_NWORKERS},
\emph{numactl} memory and cpu interleaving ~\cite{NUMA-MAN}.

To remove anamolies from experiment readings, we ran the experiment 7
times, removed the best and worst readings and then reported the
average of the remaining 5 readings.


%File Access Policy: Explain?

%We aim to add policy based file encryption scheme that prevents
%illegal access to encrypted file in case of a multi user environment.
%Our design goals are as follows.
%\begin{itemize}[leftmargin=*]
%\item \textbf{UID based Authorization}
%eCryptfs should allow a set of authorized users to access and
%transparently encrypt or decrypt files.  eCryptfs should allow
%administrators to add or update a user to the set of authorized users.
%\item \textbf{Revoke Access}
%eCryptfs should be able to revoke access for a user legally, but also
%making sure that there must be at least 1 valid user per file (do not
%revoke all of them).
%\end{itemize}
%
%%\paragraph{eCryptfs before changes}
%
%\paragraph{eCryptfs before changes}
%The Figure~\ref{fig:eCryptfs_before_changes} shows existing
%functionality of eCryptfs.  eCryptfs is a stacked file system and has
%mapping to VFS objects and operations.
%\paragraph{VFS Objects}
%eCryptfs maintains a reference between objects of lower file system
%and objects in eCryptfs file system.  The reference is maintained via
%a set of objects like
%\begin{itemize}[leftmargin=*]
%\item \emph{private\_data} of file object
%\item \emph{u.generic\_ip} of inode object
%\item \emph{d\_fsdata} of dentry object
%\item \emph{s\_fs\_info} of superblock object
%\end{itemize}
%The \emph{inode u.generic\_ip} have pointer to \emph{struct
%ecryptfs\_crypt\_stat} which contains file crypto header information.
%The crypto header is stored persistently along with file data on disk.
%
%\paragraph{VFS Operations}
%At mount time, eCryptfs-utils generates an authentication token for
%the passphrase specified by the user.  These tokens are stored in
%Linux kernel keyring and are used to setup crypto context for the
%eCryptfs files.  VFS operations from all the users just try to
%validate the file crypt headers and the authentication token.  If the
%key identifier in the header is matched against the mount-wide key
%identifier, the request is passed to the lower file system to perform
%actual IO.  We can see here that there is no support for per user
%access check , access revoke and expiry in eCryptfs kernel.  We have
%tried to address some of these limitations in our design.
%
%% Insert struct code
%\lstset{frame=l,
%  language=C,
%%numbers=left,
%%stepnumber=1,
%%numbersep=5pt,
%  showspaces=false,
%  showstringspaces=false,
%  showtabs=false,
%  tabsize=2,
%  keepspaces=true,
%  captionpos=b,
%%  breaklines=true,
%  basicstyle=\ttfamily
%%breakatwhitespace=true,
%}
%\begin{consistifyoff}
%\begin{lstlisting} [float,floatplacement=h,label=allowed_list,caption={ecryptfs\_allowed\_list structure}]
%struct ecryptfs_allowed_list {
%  uid_t a_uid[4];   /*allowed UID*/
%  uid_t a_gid[4];   /*allowed GID*/
%  uid_t a_suid[4];  /*saved UID*/
%  uid_t a_sgid[4];  /*saved GID*/
%  uid_t a_euid[4];  /*effective UID*/
%  uid_t a_egid[4];  /*effective GID*/
%};
%\end{lstlisting}
%
%\begin{lstlisting} [float=*,floatplacement=H,label=mount_crypt_stat, caption={ecryptfs\_mount\_crypt\_stat\ structure}]
%struct ecryptfs_mount_crypt_stat {
%  u32 flags;
%  struct list_head global_auth_tok_list;
%  struct mutex global_auth_tok_list_mutex;
%  size_t global_default_cipher_key_size;
%  size_t global_default_fn_cipher_key_bytes;
%  unsigned char global_default_cipher_name[ECRYPTFS_MAX_CIPHER_NAME_SIZE + 1];
%  unsigned char global_default_fn_cipher_name[ECRYPTFS_MAX_CIPHER_NAME_SIZE + 1];
%  char global_default_fnek_sig[ECRYPTFS_SIG_SIZE_HEX + 1]; 
%  struct ecryptfs_allowed_list alist;
%};
%\end{lstlisting}
%
%% Insert code.
%\begin{lstlisting} [float=*,floatplacement=H,label=ecryptfs_permission, caption={ecryptfs\_permission function}]
%static int
%ecryptfs_permission(struct inode *inode, int mask)
%{
%  struct ecryptfs_mount_crypt_stat *mount_crypt_stat;
%  struct ecryptfs_allowed_list *alist;
%  uid_t cuid;
%  mount_crypt_stat = &ecryptfs_superblock_to_private(inode->i_sb)->mount_crypt_stat;
%  cuid = current_uid().val;
%  if (mount_crypt_stat) {
%    alist = &mount_crypt_stat->alist;
%    if (cuid != alist->a_uid[0] && cuid != alist->a_uid[1] &&
%        cuid != alist->a_uid[2] && cuid != alist->a_uid[3]) {
%      return -EPERM;
%    }    
%  }
%  return inode_permission(ecryptfs_inode_to_lower(inode), mask);
%}
%\end{lstlisting}
%\end{consistifyoff}
%%\paragraph{eCryptfs after changes}
%\begin{figure*}[t]
%\centering
%\begin{minipage}[b]{0.45\textwidth}
%	\includegraphics[width=\linewidth]{figures/ecryptfs_before.eps}
%    \caption{eCryptfs before changes}
%    \label{fig:eCryptfs_before_changes}
%\end{minipage}
%\hspace{0.05\textwidth}
%\begin{minipage}[b]{0.45\textwidth}
%	\includegraphics[width=\linewidth]{figures/ecryptfs_after.eps}
%    \caption{eCryptfs after changes}
%    \label{fig:eCryptfs_after_changes}
%\end{minipage}
%\end{figure*}
%
%\paragraph{eCryptfs after changes} eCryptfs stores the mount wide
%options in \emph{struct ecryptfs\_mount\_crypt\_stat}, see
%Listing~\ref{mount_crypt_stat}.  That is stored via private field of
%superblock and every encrypted file has this structure written in the
%crypto header.  We have tried to leverage this structure to achieve
%our design goal.
%
%The Figure~\ref{fig:eCryptfs_after_changes} shows the functionality
%with our proposed changes to eCryptfs.
%
%\paragraph{Check UID Module}
%We have added a UID based security enforcement module to eCryptfs
%kernel.  This module is responsible to handle the per-user meta-data
%and access enforcements as well as file access policy management
%activities.  We added a new field in
%\emph{ecryptfs\_mount\_crypt\_stat} structure named as
%\emph{ecryptfs\_allowed\_list}, see Listing~\ref{allowed_list}.  This
%field represents the set of authorized users who have access to files.
%This field is populated at mount time and is configurable via a set of
%IOCTL commands.  First member denotes admin role, for e.g., a\_uid[0],
%a\_gid[0] are designated admin user/group and so on.  Admin roles are
%non-revocable but they can grant/revoke access to other users/groups.
%As of now we are just using \emph{a\_uid} list, but other fields can
%also be used to enforce restriction policies.
%
%\paragraph{eCryptfs Operations for UID checks}
%For every file operation VFS checks for inode permissions, which is
%passed via \emph{ecryptfs\_permission} function.  In
%\emph{ecryptfs\_permission} we have added a check if the request is
%from one of the authorized users, see
%Listing~\ref{ecryptfs_permission}.  If the user is not present in the
%set of authorized users, then return permission denied error.
%
%\paragraph{IOCTL Interface}
%We have added a set of ioctl commands to eCryptfs kernel, which can be
%used by the eCryptfs administrator to perform user policy management.
%Administrative role is assigned to an user with uid \mbox{1234}.  As
%of now admin role is fixed and non-revocable.  At the time of eCryptfs
%mount, admin user is added to the list of allowed users.  Admin user
%can use the ioctl interface to grant/revoke eCryptfs access to an
%user.  Access to eCryptfs ioctl commands is restricted to admin user
%only.  It returns permission denied if the command is raised by a
%non-admin user.
%\begin{itemize}[leftmargin=*]
%\item
%\emph{ecryptfs\_list\_users()} used to get the current allowed users
%for an eCryptfs mount.  This command returns the list of allowed UIDs.
%UID of admin user is not listed for security purposes.\\
%Usage: \emph{ecryptfs\_list\_users mount\_dir}
%\item
%\emph{ecryptfs\_allow\_user()} used to add an user to the set of
%allowed users for an eCryptfs mount.  This command takes an UID and
%tries to add it to the list of allowed users.  If the list already
%have the max number of allowed users, it returns \emph{EUSERS}.  For
%our experiment we have set the size of allowed users list to \emph{3}.
%It does takes care of already existing user in the allowed list to
%avoid duplicates.\\
%Usage: \emph{ecryptfs\_allow\_user mount\_dir UID}
%\item
%\emph{ecryptfs\_revoke\_user()} used to revoke an user's access to an
%eCryptfs mount.  This command takes an UID and tries to remove it from
%the list of allowed users.  If the user exists in the allowed user
%list, it removes the UID from the allowed list, else just return.\\
%Usage: \emph{ecryptfs\_revoke\_user mount\_dir UID}
%\end{itemize}
%
%These IOCTL commands land into eCryptfs kernel, and update the
%structure \emph{ecryptfs\_allowed\_list} from
%\emph{mount\_crypt\_stat} see Listing~\ref{mount_crypt_stat}.  There
%is a possible race condition between ioctl updating
%\emph{ecryptfs\_allowed\_list} and \emph{ecryptfs\_permisssion}
%reading the list to determine file access for a user.  Its not a
%disruptive scenario, in worst case we may transiently allow an
%operation from a revoked user.  We are planning to fix this in future
%using a lock or RCU.  \emph{mount\_crypt\_stats} is not persistent
%across mount/reboot as its container super block itself is not
%persistent.  So we plan to save the structure
%\emph{ecryptfs\_allowed\_list} to a persistent location and read it on
%a remount.  So a flexible user access management scheme can be
%deployed using these IOCTL commands.
%

%\subsection{System Operation}
%
%Example of subsection...
%
%%-----------------------------------------------------------------------------
%\paragraph{Garbage collection}
%%
%Example of a pragraph heading.
%
%\begin{figure}[htbp] \begin{centering}
%\epsfig{file=figures/lba-ind-example.eps,angle=270,width=1.00\linewidth}
%\caption{LBA indirection example.  The host first writes LBAs 23,
%352, 53, 63, 64, 65, 52, and 29.  The second write sequence is 75,
%76, 23, 52, 324, 263, and 636.  This causes LBAs 23 and 52 to become
%garbage.  Moreover, reading LBAs sequentially requires reading ABAs
%randomly.} \label{fig:lba-ind} \end{centering} \end{figure}
%
%
%Here's how you refer to Figure~\ref{fig:lba-ind}.
%
%\begin{itemize}
%
%\item itemized list item 1
%
%\item item 2
%
%\end{itemize}
%
%
%\textbf{* TENSE USE: past, present, and future}
%
%By default, everything should be written in PAST tense.  "We designed
%a system and evaluated it."
%
%Use present tense ONLY to describe system operation.  "Our system
%sends a message to the server."
%
%Use future tense ONLY in "future work" section!


%%%%%%%%%%%%%%%%%%%%%%%%%%%%%%%%%%%%%%%%%%%%%%%%%%%%%%%%%%%%%%%%%%%%%%%%%%%%%%
%% For Emacs:
% Local variables:
% fill-column: 70
% End:
%%%%%%%%%%%%%%%%%%%%%%%%%%%%%%%%%%%%%%%%%%%%%%%%%%%%%%%%%%%%%%%%%%%%%%%%%%%%%%
%% For Vim:
% vim:textwidth=70
%%%%%%%%%%%%%%%%%%%%%%%%%%%%%%%%%%%%%%%%%%%%%%%%%%%%%%%%%%%%%%%%%%%%%%%%%%%%%%
% LocalWords:
