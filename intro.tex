\section{Introduction}
\label{intro}

Breadth-first search (BFS) is being used in basically anything that
optimizes for minimum distance in some way.  It is an important kernel
used by many graph-processing applications including GPS based
navigation systems, Social network analysis tools etc.\newline
%islso known as BFS, finds shortest paths from a given source node to
%all other nodes, in terms of the number of edges in the path. It
%starts at the source node and explores the neighbour nodes first
%before moving to the next level neighbours. The traversal technique
%it is being used in basically anything that optimizes for minimum
%distance in some way.  It is an important kernel used by many
%graph-processing applications including GPS based navigation systems,
%Social network analysis tools etc.\newline
In recent years hardware and storage have become cheaper. So to scale
up we deploy multi-cores and heterogeneous computing units on a large
scale. This though speeds up the execution, leads to an increase in
power consumption and hence the maintenance costs. Hence, now a days
energy efficiency of an algorithm is becoming a matter of primary
concern.
\newline
Due to BFS traversal being so commonly used and given
the vast size of graph data sets, it alone accounts for significant
time and energy consumption these days.  This motivated us to explore
energy efficiency of existing BFS traversal techniques, analyse
results and come out with programming techniques which makes it more
energy efficient.
\newline
%Some of the key factors impacting performance are locality, cache
%efficiency, energy and power.  Graph algorithms are becoming
%increasingly important, with applications covering a wide range of
%scales. Computers run graph algorithms that reason about vast amounts
%of data, with applications including analytics and recommendation
%systems. On mobile clients, graph algorithms are important components
%of recognition and machine learning applications.  Typical length:
%1-1.5 pages.
%one-page summary of entire paper: - same 4 steps as abstract, but 1
%pgf each, instead of 1 sentence.  - intro very important: most
%reviewers make up their mind early on.
%When to write intro: - last: to ensure it properly summarizes entire
%paper.  - first: useful as an outline for rest of work
%Often best to write an outline of whole paper/ideas.
The rest of this document is organized as follows.  Section 2
describes background on parallel BFS.  Section 3 describes our
benchmarking setup.  Section 4 describes evaluation plan and results.
Section 5 describes conclusion and future work.

%%%%%%%%%%%%%%%%%%%%%%%%%%%%%%%%%%%%%%%%%%%%%%%%%%%%%%%%%%%%%%%%%%%%%
%% For Emacs:
% Local variables:
% fill-column: 70
% End:
%%%%%%%%%%%%%%%%%%%%%%%%%%%%%%%%%%%%%%%%%%%%%%%%%%%%%%%%%%%%%%%%%%%%%
%% For Vim:
% vim:textwidth=70
%%%%%%%%%%%%%%%%%%%%%%%%%%%%%%%%%%%%%%%%%%%%%%%%%%%%%%%%%%%%%%%%%%%%%%
% LocalWords:
