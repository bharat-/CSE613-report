\section{Introduction}
\label{intro}

eCryptfs~\cite{halcrow2007ecryptfs} is a POSIX-compliant
enterprise-class stacked cryptographic file system for Linux.  It is
derived from Erez Zadok's Cryptfs~\cite{zadok:cryptfs}, implemented
through the FiST framework for generating stacked file systems.
eCryptfs is a native Linux file system.  It builds as a stand-alone
kernel module for the Linux kernel; there is no need to apply any
kernel patches.  It is available in the mainline Linux Kernel as of
2.6.19.  eCryptfs is widely used, as the basis for Ubuntu's Encrypted
Home Directory, natively within Google's ChromeOS, and transparently
embedded in several network attached storage (NAS) devices.

eCryptfs stores cryptographic meta data in the header of each file, so
that encrypted files can be copied between hosts.  The file will be
decrypted only if a valid key is produced.  There is no need to keep
track of any additional information aside from what is already in the
encrypted file itself.

eCryptfs simply requires that a File Encryption Key (FEK) be
associated with any given inode in order to decrypt the contents of
the file on disk.  This prevents an attacker from accessing the file
contents outside the context of the trusted host environment.  For
instance, storage media is lost or stolen.  This is the only type of
unauthorized access that eCryptfs is intended to prevent.

But in a multiuser environment, once a user have access to the
encrypted data, there is no prevention from another user accessing the
data even if that user does not have valid key.  To tackle this
problem there are many access control mechanisms, but it is very
difficult to deny access to a root user via access control mechanisms.
Thus all the users with sudo permission can also access the data that
a user does not want to share.  eCryptfs offers no additional access
control functions other than what is already implemented via standard
POSIX file permissions, access control mechanisms (capabilities, SE
Linux) and so forth.

We have introduced a policy based authentication mechanism inside
eCryptfs.  It is an additional check to prevent unauthorized users
inside a trusted host environment from accessing the encrypted data,
even root user is not allowed.  Based on our experimental results we
see that this mechanism does not add much overhead in the current
performance of eCryptfs.

%Typical length: 1-1.5 pages.

%Intro text with a citation~\cite{cryptfs}.

%one-page summary of entire paper: - same 4 steps as abstract, but 1
%pgf each, instead of 1 sentence.  - intro very important: most
%reviewers make up their mind early on.
%
%When to write intro: - last: to ensure it properly summarizes entire
%paper.  - first: useful as an outline for rest of work
%
%Often best to write an outline of whole paper/ideas.

The rest of this document is organized as follows.  Section 2
describes the background and current limitations of eCryptfs.  Section
3 describes our proposed design.  Section 4 describes evaluation plan
and results.  Section 5 describes related work.  Section 6 describes
how the solution can be extended for different other methods,
conclusion and future work.

%%%%%%%%%%%%%%%%%%%%%%%%%%%%%%%%%%%%%%%%%%%%%%%%%%%%%%%%%%%%%%%%%%%%%
%% For Emacs:
% Local variables:
% fill-column: 70
% End:
%%%%%%%%%%%%%%%%%%%%%%%%%%%%%%%%%%%%%%%%%%%%%%%%%%%%%%%%%%%%%%%%%%%%%
%% For Vim:
% vim:textwidth=70
%%%%%%%%%%%%%%%%%%%%%%%%%%%%%%%%%%%%%%%%%%%%%%%%%%%%%%%%%%%%%%%%%%%%%%
% LocalWords:
