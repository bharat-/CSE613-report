\section{Background}
\label{bg}

%Background...
%
%Typical length: 0 pages to 1.0.
%
%Background and Related Work can be similar.  Most citations will be
%in this section.
%
%1. Describe past work and criticize it, fairly.  Use citations to
%JUSTIFY your criticism!  Problem: hard to compare to YOUR work, b/c
%you've not yet described your work in enough detail.  Solution: move
%this text to Related Work at end of paper.
%
%2. Describe in some detail, background material necessary to
%understand the rest of the paper.  Doesn't happen often, esp. if
%you've covered it in Intro.
%
%Example, submit a paper to a storage conference: reviewers are
%experts in storage.  Don't need to tell them about basic disk
%operation.  But if your paper, say, is an improvement over an
%already-advanced data structure (eg., COLA), then it'd make sense to
%describe basic COLA algorithms in some detail.
%
%Important: open the bg section with some "intro" text to tell reader
%what to expect (so experienced readers can skip it).
%
%If your bg material is too short, can fold it into opening of
%'design' section.

eCryptfs protects data confidentiality even when an unauthorized agent
gains access to the host machine.  A secret paraphrase is used to
control access to the file contents.  Properties like file name, size
and other associated meta-data are left unencrypted.  Crypto
information for each file is stored as crypt header following the
%rfc2240~\cite{rfc2240} format  along with eCryptfs marker in the file
itself, so files are portable in the event of a system or disk
failure.  Even host and persistent storage is not trusted here.

\paragraph{Current Limitations}
\begin{itemize}
\item Any user can decrypt the file, if a valid key is produced.  Once
	the file is decrypted, anyone on that host can read or write
	based on Unix permissions.  Such behavior is not suitable for
	a multi-user or file sharing environment.
\item Anyone with file permissions can delete the file.  Even if the
	file is encrypted any user with Unix permissions can modify
	the file, it may corrupt the crypto headers, hence the file
	cannot be decrypted.
\item Access revocation does not work without unmounting the eCryptfs
	file system, making it a disruptive operation.
\item There is no support for key expiry in eCryptfs.  eCryptfs can
	leverage the key expiration feature of Linux
%	kernel~\cite{keyexpire}.
\end{itemize}

%%%%%%%%%%%%%%%%%%%%%%%%%%%%%%%%%%%%%%%%%%%%%%%%%%%%%%%%%%%%%%%%%%%%%%%%%%%%%%
%% For Emacs:
% Local variables:
% fill-column: 70
% End:
%%%%%%%%%%%%%%%%%%%%%%%%%%%%%%%%%%%%%%%%%%%%%%%%%%%%%%%%%%%%%%%%%%%%%%%%%%%%%%
%% For Vim:
% vim:textwidth=70
%%%%%%%%%%%%%%%%%%%%%%%%%%%%%%%%%%%%%%%%%%%%%%%%%%%%%%%%%%%%%%%%%%%%%%%%%%%%%%
% LocalWords:
