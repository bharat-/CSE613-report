\section{Background}
\label{bg}

%Background...
%
%Typical length: 0 pages to 1.0.
%
%Background and Related Work can be similar.  Most citations will be
%in this section.
%
%1. Describe past work and criticize it, fairly.  Use citations to
%JUSTIFY your criticism!  Problem: hard to compare to YOUR work, b/c
%you've not yet described your work in enough detail.  Solution: move
%this text to Related Work at end of paper.
%
%2. Describe in some detail, background material necessary to
%understand the rest of the paper.  Doesn't happen often, esp. if
%you've covered it in Intro.
%
%Example, submit a paper to a storage conference: reviewers are
%experts in storage.  Don't need to tell them about basic disk
%operation.  But if your paper, say, is an improvement over an
%already-advanced data structure (eg., COLA), then it'd make sense to
%describe basic COLA algorithms in some detail.
%
%Important: open the bg section with some "intro" text to tell reader
%what to expect (so experienced readers can skip it).
%
%If your bg material is too short, can fold it into opening of
%'design' section.

BFS is an algorithm for traversing or searching tree or graph data structures.
It starts at some arbitrary node of a graph, sometimes referred to as source
vertex and explores the neighbor nodes first, before moving to the next level
neighbors.

Its a well known research topic, a variety of parallel BFS algorithms have
since been explored trying to improve complexity, parallelism, distributed,
cache and io performance. Some of these parallel algorithms are \emph{work
efficient}, meaning that the total number of operations performed is the same
to within a constant factor as that of a comparable serial algorithm. 

Very few have worked to optimize energy/power efficiency of BFS algorithm.
Considering the recent trends in application of graph algorithms in big data
and information analysis, energy efficiency is a major concern.

In this project we have analysed some of the existing parallel BFS algorithms
in terms of energy efficiency.


%%%%%%%%%%%%%%%%%%%%%%%%%%%%%%%%%%%%%%%%%%%%%%%%%%%%%%%%%%%%%%%%%%%%%%%%%%%%%%
%% For Emacs:
% Local variables:
% fill-column: 70
% End:
%%%%%%%%%%%%%%%%%%%%%%%%%%%%%%%%%%%%%%%%%%%%%%%%%%%%%%%%%%%%%%%%%%%%%%%%%%%%%%
%% For Vim:
% vim:textwidth=70
%%%%%%%%%%%%%%%%%%%%%%%%%%%%%%%%%%%%%%%%%%%%%%%%%%%%%%%%%%%%%%%%%%%%%%%%%%%%%%
% LocalWords:
