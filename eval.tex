\section{Evaluation}
\label{eval}

%Evaluation section...
%
%Length: 3-4 pages (graphs take a lot of space!)
%
%Tell reader what to expect.
%
%Eval is "proof" that your design is good.
%MATCH eval goals with design goals.
%
%Eval section is easier to write, but longest one to produce
%results for.  Whereas design section has complex structure, eval
%is more 'flat'.
%
%Structure:
%
%1.  list eval goals, should match design goals
%
%2.  briefly list how you plan to prove those goals.
%
%3.  describe your testbed: h/w + s/w platform to run tests on.
%Give enough detail so it can be reproduced by ANYONE.
%
%4.  describe your benchmarks in detail:
%
%(4a) Micro benchmarks: test specific feature (e.g., read or
%write performance).  Usually u-bench are designed to highlight
%worst/best case behavior of your system.  Be to list both best
%and worst.
%
%(4b) Macro benchmarks (general purpose benchmarks): test whole
%system (e.g., run a Web server exerciser, or TPC for database).
%
%Some tests should compare YOUR system to past systems, or a
%"before and after" comparison.
%
%For every possible variable in your system, design a set of
%independent tests (re: compression study's dimensions).  Justify
%need to vary each variable (the more variables, the more
%experiments you have to run).
%
%5.  Describe your benchmarking methodology
%
%Statistical stability: how many times you run each test? do you
%compute standard deviations, half-width intervals (for student-t
%distribution), RMS, or other metric of stability?  Say how many
%times you ran each test, and what were the stability metrics.
%
%Ex."we ran every test at least 10 times, and computed the
%standard deviation as a percentage of the mean.  In all cases,
%the percentage was less than 5\%, unless otherwise noted."
%
%6.  List every benchmark result, for each test
%
%(a) a graph or table or other figure, plus caption.
%(b) followed by an explanation of the figure: say what one sees
%    in the figure, then explain WHY it is so.
%
%7.  Optional: if eval section longer than usual, end it with a
%one-paragraph summary of eval results.

%\begin{table}[th]
%\begin{center}
%    \begin{tabular}{| l | l | l | l | l | l | l |}
%    \hline
%	Dataset & Algo1 & Algo2 & Algo3 & Algo4 & Algo5 & Algo6\\ \hline
%	gre\_1107 & 1.19 & 5.17 & 0.93 & 1.35 & 1.25 & 1.33 \\ \hline
%	cell1 & 4.16 & 20.89 & 2.09 & 3.62 & 4.53 & 4.73\\ \hline
%	appu & 172.91 & 112.16 & 92.21 & 8.92 & 171.13 & 175.07\\ \hline
%	conf6 & 185.58 & 119.82 & 103.45 & 9.98 & 191.85 & 185.86\\ \hline
%	dblp & 244.31 & 148.81 & 124.70 & 13.96 & 241.30 & 235.15\\ \hline
%	amazon & 220.47 & 136.91 & 109.67 & 16.16 & 224.48 & 208.35\\ \hline
%	fem & 2091.23 & 1182.47 & 1119.56 & 87.31 & 2092.19 & 2102.01\\ \hline
%	Chevron4 & 645.25 & 481.38 & 338.31 & 76.22 & 642.99 & 640.16\\ \hline
%	cage14 & 2953.03 & 1626.01 & 1555.99 & 129.44 & 2942.22 & 2973.07\\ \hline
%	cage15 & 11267.14 & 6200.88 & 5936.38 & 499.53 & 11390.66 & 11452.49\\ \hline
%	delaunay & 12762.75 & 7124.39 & 6682.38 & 751.73 & 12893.38 & 12677.84\\ \hline
%    \hline
%    \end{tabular}
%\end{center}
%\caption{\capfont ENERGY readings}
%\label{tab:Table1}
%\end{table}
%
%
%
%\begin{table}[th]
%\begin{center}
%    \begin{tabular}{| l | l | l | l | l | l | l |}
%    \hline
%	Dataset & Algo1 & Algo2 & Algo3 & Algo4 & Algo5 & Algo6\\ \hline
%	gre & 63.82 & 67.15 & 60.99 & 82.85 & 63.11 & 66.32 \\ \hline
%	cell1 & 67.53 & 67.51 & 60.55 & 111.95 & 71.61 & 71.36\\ \hline
%	appu & 72.37 & 64.35 & 77.86 & 111.53 & 73.79 & 74.17\\ \hline
%	conf6 & 74.85 & 65.35 & 78.20 & 113.69 & 72.82 & 75.78\\ \hline
%	dblp & 76.69 & 67.59 & 80.60 & 117.85 & 76.26 & 73.07\\ \hline
%	amazon & 70.46 & 65.29 & 77.67 & 126.79 & 76.42 & 76.25\\ \hline
%	fem & 77.09 & 73.49 & 78.85 & 142.76 & 76.03 & 76.07\\ \hline
%	Chevron4 & 77.15 & 68.78 & 78.10 & 160.25 & 78.93 & 77.72\\ \hline
%	cage14 & 77.31 & 73.71 & 78.63 & 156.93 & 77.08 & 77.23\\ \hline
%	cage15 & 77.06 & 74.79 & 77.71 & 176.09 & 77.70 & 77.68\\ \hline
%	delaunay & 76.83 & 74.16 & 78.93 & 181.46 & 77.45 & 76.89\\ \hline
%    \hline
%    \end{tabular}
%\end{center}
%\caption{\capfont POWER readings}
%\label{tab:Table2}
%\end{table}


\begin{table}[th]
\begin{center}
    \begin{tabular}{| l | l | l | l | l | l | l |}
    \hline
	Dataset & Algo1 & Algo2 & Algo3 & Algo4 & Algo5 & Algo6\\ \hline
	Gre & 3.46 & 22.03 & 4.25 & 4.2 & 5.4 & 5.64 \\ \hline
	Cell1 & 10.57 & 99.92 & 3.65 & 56.92 & 14.37 & 16.79\\ \hline
	Appu & 150.85 & 152.95 & 121.79 & 445.63 & 123.45 & 130.12\\ \hline
	Conf6 & 155.97 & 153.01 & 133.67 & 473.59 & 145.70 & 145.20\\ \hline
	Dblp & 333.22 & 282.73 & 193.29 & 682.99 & 291.98 & 268.84\\ \hline
	Amazon & 322.82 & 317.66 & 173.32 & 699.97 & 246.82 & 255.30\\ \hline
	Fem & 1607.10 & 2264.73 & 1398.64 & 6774.77 & 1616.92 & 1573.88\\ \hline
	Chevron & 697.41 & 920.61 & 577.01 & 3145.14 & 695.26 & 764.84\\ \hline
	Cage14 & 2830.57 & 3837.75 & 2060.95 & 11183.9 & 2876.8 & 2887.03\\ \hline
	Cage15 & 10850.6 & 15121.6 & 8351.24 & 44794.6 & 11106.5 & 11088.2\\ \hline
	Delaunay & 14667.8 & 13353.9 & 9447.41 & 52910.3 & 13614.7 & 13194.1\\ \hline
    \hline
    \end{tabular}
\end{center}
\caption{\capfont MEM readings in MB}
\label{tab:Table3}
\end{table}


\begin{table*}[th]
\small
\centering
%\begin{tabularx}{\linewidth}{|c|c|c|c|c|c|c|X|}

\begin{tabular}{ c|c|c|c|c|c|c|c|c|c|c|c|c| }
  \cline{2-13}
&
\multicolumn{6}{c}{\textbf{ENERGY}}&
  \multicolumn{6}{|c}{\textbf{POWER}} \\
  \cline{2-13}
  \multicolumn{1}{c|}{} &
  Algo1 & Algo2 & Algo3 & Algo4 & Algo5 & Algo6 & Algo1 & Algo2 & Algo3 & Algo4 & Algo5 & Algo6\\\hline
  \hline
  \multicolumn{1}{|c|}{\textbf{gre}}
& 1.19 & 5.17 & 0.93 & 1.35 & 1.25 & 1.33 & 63.82 & 67.15 & 60.99 & 82.85 & 63.11 & 66.32 \\ \hline
  \hline
  \multicolumn{1}{|c|}{\textbf{cell1}}
& 4.16 & 20.89 & 2.09 & 3.62 & 4.53 & 4.73 & 67.53 & 67.51 & 60.55 & 111.95 & 71.61 & 71.36\\ \hline
  \hline
  \multicolumn{1}{|c|}{\textbf{appu}}
& 172.91 & 112.16 & 92.21 & 8.92 & 171.13 & 175.07 & 72.37 & 64.35 & 77.86 & 111.53 & 73.79 & 74.17\\ \hline
  \hline
  \multicolumn{1}{|c|}{\textbf{conf6}}
& 185.58 & 119.82 & 103.45 & 9.98 & 191.85 & 185.86 & 74.85 & 65.35 & 78.20 & 113.69 & 72.82 & 75.78\\ \hline
  \hline
  \multicolumn{1}{|c|}{\textbf{dblp}}
& 244.31 & 148.81 & 124.70 & 13.96 & 241.30 & 235.15 & 76.69 & 67.59 & 80.60 & 117.85 & 76.26 & 73.07\\ \hline
  \hline
  \multicolumn{1}{|c|}{\textbf{amazon}}
& 220.47 & 136.91 & 109.67 & 16.16 & 224.48 & 208.35 & 70.46 & 65.29 & 77.67 & 126.79 & 76.42 & 76.25\\ \hline
  \hline
  \multicolumn{1}{|c|}{\textbf{fem}}
& 2091.23 & 1182.47 & 1119.56 & 87.31 & 2092.19 & 2102.01 & 77.09 & 73.49 & 78.85 & 142.76 & 76.03 & 76.07\\ \hline
  \hline
  \multicolumn{1}{|c|}{\textbf{Chevron4}}
& 645.25 & 481.38 & 338.31 & 76.22 & 642.99 & 640.16 & 77.15 & 68.78 & 78.10 & 160.25 & 78.93 & 77.72\\ \hline
  \hline
  \multicolumn{1}{|c|}{\textbf{cage14}}
& 2953.03 & 1626.01 & 1555.99 & 129.44 & 2942.22 & 2973.07 & 77.31 & 73.71 & 78.63 & 156.93 & 77.08 & 77.23\\ \hline
  \hline
  \multicolumn{1}{|c|}{\textbf{cage15}}
& 11267.14 & 6200.88 & 5936.38 & 499.53 & 11390.66 & 11452.49 & 77.06 & 74.79 & 77.71 & 176.09 & 77.70 & 77.68\\ \hline
  \hline
  \multicolumn{1}{|c|}{\textbf{delaunay}}
& 12762.75 & 7124.39 & 6682.38 & 751.73 & 12893.38 & 12677.84 & 76.83 & 74.16 & 78.93 & 181.46 & 77.45 & 76.89\\ \hline
  \hline
\end{tabular}

%\end{tabularx}
\caption{\capfont ENERGY (in Joules) and POWER (in Watts) readings }
\label{tab:table1}
\end{table*}


\begin{table*}[th]
\small
\centering
%\begin{tabularx}{\linewidth}{|c|c|c|c|c|c|c|X|}

\begin{tabular}{ c|c|c|c|c|c|c|c|c|c|c|c|c| }
  \cline{2-13}
&
\multicolumn{6}{c}{\textbf{L2CACHE}} &
  \multicolumn{6}{|c}{\textbf{L3CACHE}} \\
  \cline{2-13}
  \multicolumn{1}{c|}{} &
  Algo1 & Algo2 & Algo3 & Algo4 & Algo5 & Algo6 & Algo1 & Algo2 & Algo3 & Algo4 & Algo5 & Algo6\\\hline
  \hline
  \multicolumn{1}{|c|}{\textbf{gre}}
& 4.55 & 4.03 & 4.81 & 3.45 & 3.57 & 3.85 & 4.66 & 5.07 & 4.06 & 7.46 & 5.12 & 4.96 \\ \hline
  \hline
  \multicolumn{1}{|c|}{\textbf{cell1}}
& 4.69 & 3.73 & 3.66 & 4.98 & 4.78 & 4.17 & 6.12 & 4.67 & 4.02 & 13.97 & 7.66 & 7.67\\ \hline
  \hline
  \multicolumn{1}{|c|}{\textbf{appu}}
& 3.54 & 5.10 & 3.84 & 3.49 & 4.31 & 4.54 & 4.51 & 4.61 & 7.73 & 10.13 & 5.38 & 5.29\\ \hline
  \hline
\multicolumn{1}{|c|}{\textbf{conf6}}
& 4.99 & 4.14 & 4.67 & 3.81 & 5.07 & 4.49 & 5.16 & 5.77 & 8.58 & 10.94 & 5.57 & 5.47\\ \hline
  \multicolumn{1}{|c|}{\textbf{dblp}}
& 4.68 & 4.25 & 5.11 & 3.92 & 5.03 & 4.65 & 4.77 & 9.65 & 9.34 & 11.30 & 7.18 & 7.18\\ \hline
  \hline
  \multicolumn{1}{|c|}{\textbf{amazon}}
& 4.70 & 4.39 & 5.07 & 3.35 & 4.98 & 4.54 & 5.40 & 10.80 & 13.02 & 9.34 & 8.38 & 8.65\\ \hline
  \hline
  \multicolumn{1}{|c|}{\textbf{fem}}
& 4.83 & 4.15 & 4.62 & 3.97 & 4.92 & 4.67 & 5.21 & 9.07 & 9.49 & 10.11 & 5.15 & 6.84\\ \hline
  \hline
  \multicolumn{1}{|c|}{\textbf{Chevron4}}
& 4.85 & 4.10 & 4.70 & 4.09 & 5.04 & 4.50 & 8.62 & 4.21 & 10.85 & 13.68 & 11.46 & 12.02\\ \hline
  \hline
  \multicolumn{1}{|c|}{\textbf{cage14}}
& 4.76 & 4.03 & 4.97 & 4.03 & 4.92 & 4.59 & 4.86 & 9.15 & 10.46 & 9.58 & 5.18 & 5.77\\ \hline
  \hline
  \multicolumn{1}{|c|}{\textbf{cage15}}
& 4.61 & 4.09 & 4.83 & 3.84 & 4.56 & 4.75 & 5.28 & 10.12 & 7.94 & 9.65 & 5.31 & 5.46\\ \hline
  \hline
  \multicolumn{1}{|c|}{\textbf{delaunay}}
& 4.36 & 4.08 & 4.69 & 4.06 & 4.34 & 4.61 & 5.84 & 16.00 & 7.63 & 6.54 & 7.63 & 8.18\\ \hline
  \hline
\end{tabular}

%\end{tabularx}
\caption{\capfont L2CACHE and L3CACHE miss ratio readings }
\label{tab:table2}
\end{table*}





























%%%%%%%%%%%%%%%%%%%%%%%%%%%%%%%%%%%%%%%%%%%%%%%%%%%%%%%%%%%%%%%%%%%%%%%%%%%%%%%
% Table(s) for results of current experiments
%%%%%%%%%%%%%%%%%%%%%%%%%%%%%%%%%%%%%%%%%%%%%%%%%%%%%%%%%%%%%%%%%%%%%%%%%%%%%%

\begin{table*}[th]
\small
\centering
%\begin{tabularx}{\linewidth}{|c|c|c|c|c|c|c|X|}

\begin{tabular}{ c|c|c|c|c|c|c|c|c| }
  \cline{2-9}
  & 
  \multicolumn{2}{c}{\textbf{$u_{1}$}} &
  \multicolumn{2}{|c}{\textbf{$u_{2}$}} &
  \multicolumn{2}{|c}{\textbf{$u_{3}$}} &
  \multicolumn{2}{|c|}{\textbf{root}}  \\
  \cline{2-9}
  \multicolumn{1}{c|}{} &
  with & without & with & without & with & without & with & without\\
  \multicolumn{1}{c|}{} &
  change & changes & changes & changes & changes & changes & changes &
  changes\\
  \hline
  \multicolumn{1}{|c|}{\textbf{create file}}
  & Y & Y & N & Y & N & Y & N & Y \\
  \hline
  \multicolumn{1}{|c|}{\textbf{create dir}}
  & Y & Y & N & Y & N & Y & N & Y \\
  \hline
  \multicolumn{1}{|c|}{\textbf{remove file}}
  & Y & Y & N & Y & N & Y & N & Y \\
  \hline
  \multicolumn{1}{|c|}{\textbf{remove dir}}
  & Y & Y & N & Y & N & Y & N & Y \\
  \hline
  \multicolumn{1}{|c|}{\textbf{create symlink}}
  & Y & Y & N & Y & N & Y & N & Y \\
  \hline
  \multicolumn{1}{|c|}{\textbf{read symlink}}
  & Y & Y & N & Y & N & Y & N & Y \\
  \hline
  \multicolumn{1}{|c|}{\textbf{write symlink}}
  & Y & Y & N & Y & N & Y & N & Y \\
  \hline
  \multicolumn{1}{|c|}{\textbf{create hardlink}}
  & Y & Y & N & Y & N & Y & N & Y \\
  \hline
  \multicolumn{1}{|c|}{\textbf{write hardlink}}
  & Y & Y & N & Y & N & Y & N & Y \\
  \hline
  \multicolumn{1}{|c|}{\textbf{stat}}
  & Y & Y & N & Y & N & Y & N & Y \\
  \hline
  \multicolumn{1}{|c|}{\textbf{change dir}}
  & Y & Y & N & Y & N & Y & N & Y \\
  \hline
  \multicolumn{1}{|c|}{\textbf{read file}}
  & Y & Y & N & Y & N & Y & N & Y \\
  \hline
  \multicolumn{1}{|c|}{\textbf{write file}}
  & Y & Y & N & Y & N & Y & N & Y \\
  \hline
  \multicolumn{1}{|c|}{\textbf{create tar}}
  & Y & Y & N & Y & N & Y & N & Y \\
  \hline
  \multicolumn{1}{|c|}{\textbf{untar}}
  & Y & Y & N & Y & N & Y & N & Y \\
  \hline
  \multicolumn{1}{|c|}{\textbf{make}}
  & Y & Y & N & Y & N & Y & N & Y \\
  \hline
  \multicolumn{1}{|c|}{\textbf{rename}}
  & Y & Y & N & Y & N & Y & N & Y \\
  \hline
\end{tabular}

%\end{tabularx}
\caption{\capfont Results of different file system operations for
different users, with and without the changes.}
\label{tab:results}
\end{table*}

%%%%%%%%%%%%%%%%%%%%%%%%%%%%%%%%%%%%%%%%%%%%%%%%%%%%%%%%%%%%%%%%%%%%%%%%%%%%%%
%% For Emacs:
% Local variables:
% fill-column: 70
% End:
%%%%%%%%%%%%%%%%%%%%%%%%%%%%%%%%%%%%%%%%%%%%%%%%%%%%%%%%%%%%%%%%%%%%%%%%%%%%%%
%% For Vim:
% vim:textwidth=70
%%%%%%%%%%%%%%%%%%%%%%%%%%%%%%%%%%%%%%%%%%%%%%%%%%%%%%%%%%%%%%%%%%%%%%%%%%%%%%
% LocalWords:  PEAFS PEAIO Lustre SBU HMC config

%%%%%%%%%%%%%%%%%%%%%%%%%%%%%%%%%%%%%%%%%%%%%%%%%%%%%%%%%%%%%%%%%%%%%%%%%%%%%%%
% Table(s) for results of current experiments
%%%%%%%%%%%%%%%%%%%%%%%%%%%%%%%%%%%%%%%%%%%%%%%%%%%%%%%%%%%%%%%%%%%%%%%%%%%%%%

\begin{table*}[th]
\small
\centering
%\begin{tabularx}{\linewidth}{|c|c|c|c|c|c|c|X|}

\begin{tabular}{ c|c|c|c|c|c|c| }
  \cline{2-7}
  & 
  \multicolumn{2}{c}{\textbf{root}} &
  \multicolumn{2}{|c}{\textbf{user1}} &
  \multicolumn{2}{|c|}{\textbf{root\_notallowed}}\\ 
  \cline{2-7}
  \multicolumn{1}{c|}{} &
  with & without & with & without & with & without\\
  \multicolumn{1}{c|}{} &
  changes & changes & changes & changes & changes & changes\\
  \hline
  \multicolumn{1}{|c|}{\textbf{xfstests}}
  & 56(68) & 56(68) & 32(68) & 32(68) & 0(68) & - \\
  \hline
  \multicolumn{1}{|c|}{\textbf{eCryptfs-tests}}
  & 25(25) & 25(25) & 22(25) & 22(25) & 0(25) & - \\
  \hline
\end{tabular}

%\end{tabularx}
\caption{\capfont Number of passed tests and total tests for
\emph{XFSTEST} and \emph{eCryptfs-tests} for different users, with and
without the changes.}
\label{tab:results-xfs}
\end{table*}

%%%%%%%%%%%%%%%%%%%%%%%%%%%%%%%%%%%%%%%%%%%%%%%%%%%%%%%%%%%%%%%%%%%%%%%%%%%%%%
%% For Emacs:
% Local variables:
% fill-column: 70
% End:
%%%%%%%%%%%%%%%%%%%%%%%%%%%%%%%%%%%%%%%%%%%%%%%%%%%%%%%%%%%%%%%%%%%%%%%%%%%%%%
%% For Vim:
% vim:textwidth=70
%%%%%%%%%%%%%%%%%%%%%%%%%%%%%%%%%%%%%%%%%%%%%%%%%%%%%%%%%%%%%%%%%%%%%%%%%%%%%%
% LocalWords:  PEAFS PEAIO Lustre SBU HMC config

%
%\paragraph{Evaluation Goals}
%We aimed to provide a correct working solution for eCryptfs with minimal
%performance overheads.
%
%\begin{itemize}
%\item
%\textbf{Correctness}\\
%Only the users that are allowed to access, should be able to access
%the file contents.  Other users should get a permission denied error
%irrespective of their privilege level.
%
%The user who is assigned the administrative role should be able to add
%or revoke permissions to other users in the system.
%
%Users without administrative privilege should not be able to change
%file access policy or gain illegal access.
%\item
%\textbf{Performance}\\ Performance of underlying file system should
%not suffer a penalty due to this extra security enforcement in
%eCryptfs.
%
%We keep the policy management separate from data path, so there is no
%overhead from management tasks on system performance.
%\item
%\textbf{Regression testing}\\ This change should not break any
%existing functionality in eCryptfs as well as the underlying file
%system.
%
%We have tested our changes with basic file operations like create
%file, directory, symlink, delete, rename, untar, kernel compilation.
%We also have run more comprehensive tests such as
%\emph{XFSTESTS}~\cite{xfstests} and eCryptfs-utils test scripts.
%\end{itemize}
%
%\paragraph{Evaluation Plan}
%We have run a set of testing tools along with our manually written
%tests.  The tools are \emph{XFSTESTS} and test scripts provided in
%eCriptfs-utils.  We have run these tools on both modified and
%unmodified eCryptfs kernel for different users with different
%privilege level.  We have tried to run the maximum number of tests
%from these tools, in case of a test failure we rerun it in isolation and
%identify the root cause of failure.  Test cases from these tools are
%sufficient, as they test for the functionality and regression of the file
%system.
%
%\paragraph{Experimental Setup}
%We have used a virtual machine with two Intel\textregistered
%Xeon\texttrademark dual-core 2.40GHz CPUs, and 8GB RAM.  The machine
%ran \mbox{Ubuntu} 14.04.1 with a vanilla 3.19.5 Linux Kernel.  We
%chose \mbox{Ubuntu} because it is freely available and the package
%eCryptfs-utils is installed by default.  For our experiments, we have
%4 users $u_1$, $u_2$, $u_3$ and a root user.  Only user $u_1$ is
%authorized to access encrypted files.  User $u_2$ is placed in sudoer
%list, $u_3$ is a normal user.  All users have intention to access the
%encrypted data, so we have 3 type of attackers and one authorized
%user.
%
%
%\begin{figure}[t]
%    \centering
%    \includegraphics[width=0.5\textwidth]{figures/perf-results.eps}
%    \caption{eCryptfs performance comparison with and without our
%    changes for make, tar, untar operations.}
%    \label{fig:perf-results}
%\end{figure}
%
%
%\paragraph{Results}
%We ran basic file system operation for all four users mentioned above
%on the setup described.  We made changes to Linux Kernel 3.19.5.
%Table~\ref{tab:results} describes the results for all the file systems
%operations we ran.  Column 1 describes the file system operation.
%Column 2,3,4 shows results for users $u_1$, $u_2$, $u_3$ and root user
%respectively with and without our changes in Linux kernel.  \textbf{Y}
%indicates that the  operation ran successfully.  \textbf{N} indicates
%that the user was denied to run the corresponding commands inside the
%eCryptfs mount point.  We also ran some performance tests, such as
%make a Linux Kernel, tar/untar the source tree of Linux Kernel.
%Figure~\ref{fig:perf-results} shows that there is no significant
%performance overhead due to our changes.  The overhead for make workload
%is \emph{1.17\%} and for tar/untar is \emph{1.66\%}.
%
%We ran \emph{XFSTESTS} test-suite on Linux-3.19.5 vanilla kernel for
%\emph{eCryptfs} file system.  We used \emph{FSL} git repository of
%\emph{XFSTESTS} for our tests.  We modified the test-suite to support
%\emph{eCryptfs} file system.  These tests can be categorized in two
%categories, 1, without changes and 2, with changes in eCryptfs kernel
%code.  For category 1 we ran the test-suite for root user and a
%non-root user.  For category 2 we ran the test-suite for an allowed
%non-root user, and root user.  Then we added the root user to allowed
%users list and ran the test-suite again.  We compared the test results
%for both categories for each corresponding user.  We find there is no
%difference in the results.  We see that for category 2, when root user
%was not allowed to access eCryptfs file system, all tests failed.
%
%We also found new interesting bugs with eCryptfs file system.
%\begin{itemize}
%\item 
%\textbf{noatime mount option:} If the eCryptfs file system is mounted
%with noatime mount option, the access times for any file should never
%change.  But we see that there is no difference between if file system
%is mounted with noatime option or not.  The access time for files
%changes in both the cases.
%
%\item
%\textbf{Direct IO:} eCryptfs does not support direct IO.  So all the
%tests for direct IO failed.
%\item
%\textbf{File access timestamp Epoch:} Create a file in eCryptfs
%directory with creation time before epoch.  Check the time of last
%access as seconds since Epoch, it should be a negative value.  Now
%flush the cache by remounting the file system.  Now if we check again
%for the time of last access as seconds since Epoch, it should still be
%negative, but that is not the case here.
%\end{itemize}
%
%All the \emph{XFSTESTS} that failed can be assigned to the issues
%mentioned above.
%
%We also ran the default eCryptfs-test script found in eCryptfs utils.
%These tests include various file systems tests, such as file read,
%write, concurrent access, inode races, symlinks, file truncate.  These
%tests also include tests for already identified bugs in eCryptfs.
%These tests ensures that we did not break anything that was working
%before our changes.  For both the categories, we did not find any
%difference in the results for all the tests.  As expected for any
%non-allowed user all the tests in the script failed.
%
%Table~\ref{tab:results-xfs} shows the number of tests that passed from
%total number of tests for both the test-suites for various users.  We
%examined the failed tests and identified why some of \emph{XFSTESTS}
%failed.  The numbers show that there was nothing broken due to our
%changes.  Since many tests requires root permissions as these are
%kernel level tests, the amount of tests that failed for non-root user
%were considerably higher than that of root user.
%
%IOCTL interface is restricted to admin user.  For our experiments we
%hard coded a UID \mbox{1234} as admin user in eCryptfs kernel.  We
%created a user with UID \mbox{1234} using adduser(1) utility.  At
%mount time only admin user is added to the list of allowed users.
%Root can mount the file system but cannot access the files within the
%mount point, as it is not part of allowed list.  At this point only
%admin user can perform file operation and user management activity.
%Once the admin added a user to the set of allowed users, that user can
%perform file operation.  Still user management operations like
%list\_users/allow\_user/revoke\_user is restricted to admin user.
%
%We tested IOCTL interface via multiple users like admin, user1, user2,
%root.  We observed that only admin user was able to successfully run
%the IOCTL commands, other users including root got permission denied
%error.  Since admin role is non-revocable, even the admin user cannot
%revoke itself.  Admin user cannot add any users to the allowed list
%once the list was full, too many users error is returned.  We verified
%that the file operation path was being properly affected by the
%changes done via IOCTL interface.
%
%%%%%%%%%%%%%%%%%%%%%%%%%%%%%%%%%%%%%%%%%%%%%%%%%%%%%%%%%%%%%%%%%%%%%%%%%%%%%%
%% For Emacs:
% Local variables:
% fill-column: 70
% End:
%%%%%%%%%%%%%%%%%%%%%%%%%%%%%%%%%%%%%%%%%%%%%%%%%%%%%%%%%%%%%%%%%%%%%%%%%%%%%%
%% For Vim:
% vim:textwidth=70
%%%%%%%%%%%%%%%%%%%%%%%%%%%%%%%%%%%%%%%%%%%%%%%%%%%%%%%%%%%%%%%%%%%%%%%%%%%%%%
% LocalWords:
