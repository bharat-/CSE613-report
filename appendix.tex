\section{General Purpose Notes}

\subsection{Notes About Picking a Project}

Put every possible related citation you can! (esp. if conf.
doesn't count citations towards page size).

Literature survey:
- CiteSeer

- Google Scholar

- libraries

1. find a few relates paper

2. skim papers to find relevance

3. search for add'l related papers in Biblio.

4. reverse citation: use srch engines, to find
   newer papers that cite the paper you like.

5. "stop" when reach transitive closure

- then go off and read it; summarize papers

- think about "how can I improve" and "what was so
  good about that paper".

- check future work for project ideas.

- go to talks \& conferences

Pick an idea:

- novelty vs. incremental (how big of an increment?)

- idea vs. practical implications
  (implemented? released? in use as OSS or commercial?)

- where to submit? good fit and match for quality.

- look at schedule of conferences: due dates \& result dates.
i.e., what's the due dates, when do you hear an accept/reject
notice by; if the paper is rejected, where/when can you submit
it to next?

\subsection{Different Types of Technical Papers}

Computer systems field:

- operating systems, networking, security

- programming, virtualization, storage

- architecture, databases, etc.

- NOT: theory

1. conference paper: mature, completed work. [length 6-14 pages]

2. workshop paper: short, work-in-progress report [4-6 pp]

- special workshop type: a "position" paper, often called "Hot-???"

3. journal article: expanded version of a conference/workshop
   paper.  No page limit.  Common to follow up a workshop and even
   a conf. paper with an exapnded journal version. Journals often
   ask for at least 25\% more "new" material.

Conf./workshop/journals are "refereed" publications, meaning that
your peers get to review the document and accept/reject it, and
return comments to you.  Non-refereed pubs are usually called
"technical reports": anyone can publish a TR on their own.

Workshop and conf. papers get presented in person: usually lead
author would give the talk.  Journals are not presented.

In this class, you will submit a "position" paper (your design
document, 2 pp); by end of term, you'll submit a longer conf. like'
paper (6+ pp).

Conf./workshop papers review cycle: 1-4 months

- journals: no time limit; often; no deadline often 6+ months.

Conf./workshop papers get an accept/reject notice:

- rate: a conditional accept pending "shepherd's aproval"

Journals: phases of aproval:

- reject: "go away"

- major revision: a reject; ask to make major changes and resubmit.

- minor revision: a reject; ask to make minor changes and resubmit.

- conditional accept: very minor changes asked (usually prose,
  typos, English)

- accept: manuscript is accepted as is.

Publishing cycles, from moment of receiving an "accept notice":

- workshop/conf.: 2-3 months; present @ venue 2-3 months later.

- journal: 2-6 months

A manuscript is considered published on 1st day of workshop/conf.
or the first day of the month in which the journal is published.

%%%%%%%%%%%%%%%%%%%%%%%%%%%%%%%%%%%%%%%%%%%%%%%%%%%%%%%%%%%%%%%%%%%%%%%%%%%%%%
%% For Emacs:
% Local variables:
% fill-column: 70
% End:
%%%%%%%%%%%%%%%%%%%%%%%%%%%%%%%%%%%%%%%%%%%%%%%%%%%%%%%%%%%%%%%%%%%%%%%%%%%%%%
%% For Vim:
% vim:textwidth=70
%%%%%%%%%%%%%%%%%%%%%%%%%%%%%%%%%%%%%%%%%%%%%%%%%%%%%%%%%%%%%%%%%%%%%%%%%%%%%%
% LocalWords:  SMR HDDs drive's SMRs


